%%%%%%%%%%%%%%%%%%%%%%%%%%%%%%%%%%%%%%%%%
% Short Sectioned Assignment LaTeX Template Version 1.0 (5/5/12)
% This template has been downloaded from: http://www.LaTeXTemplates.com
% Original author:  Frits Wenneker (http://www.howtotex.com)
% License: CC BY-NC-SA 3.0 (http://creativecommons.org/licenses/by-nc-sa/3.0/)
%%%%%%%%%%%%%%%%%%%%%%%%%%%%%%%%%%%%%%%%%

%----------------------------------------------------------------------------------------
%	PACKAGES AND OTHER DOCUMENT CONFIGURATIONS
%----------------------------------------------------------------------------------------

\documentclass[paper=a4, fontsize=11pt]{scrartcl} % A4 paper and 11pt font size

% ---- Entrada y salida de texto -----

\usepackage[T1]{fontenc} % Use 8-bit encoding that has 256 glyphs
\usepackage[utf8]{inputenc}
%\usepackage{fourier} % Use the Adobe Utopia font for the document - comment this line to return to the LaTeX default

% ---- Idioma --------

\usepackage[spanish, es-tabla]{babel} % Selecciona el español para palabras introducidas automáticamente, p.ej. "septiembre" en la fecha y especifica que se use la palabra Tabla en vez de Cuadro

% ---- Otros paquetes ----

\usepackage{amsmath,amsfonts,amsthm} % Math packages
%\usepackage{graphics,graphicx, floatrow} %para incluir imágenes y notas en las imágenes
\usepackage{graphics,graphicx, float} %para incluir imágenes y colocarlas

% Para hacer tablas comlejas
%\usepackage{multirow}
%\usepackage{threeparttable}

%\usepackage{sectsty} % Allows customizing section commands
%\allsectionsfont{\centering \normalfont\scshape} % Make all sections centered, the default font and small caps

\usepackage{fancyhdr} % Custom headers and footers
\pagestyle{fancyplain} % Makes all pages in the document conform to the custom headers and footers
\fancyhead{} % No page header - if you want one, create it in the same way as the footers below
\fancyfoot[L]{} % Empty left footer
\fancyfoot[C]{} % Empty center footer
\fancyfoot[R]{\thepage} % Page numbering for right footer
\renewcommand{\headrulewidth}{0pt} % Remove header underlines
\renewcommand{\footrulewidth}{0pt} % Remove footer underlines
\setlength{\headheight}{13.6pt} % Customize the height of the header

\numberwithin{equation}{section} % Number equations within sections (i.e. 1.1, 1.2, 2.1, 2.2 instead of 1, 2, 3, 4)
\numberwithin{figure}{section} % Number figures within sections (i.e. 1.1, 1.2, 2.1, 2.2 instead of 1, 2, 3, 4)
\numberwithin{table}{section} % Number tables within sections (i.e. 1.1, 1.2, 2.1, 2.2 instead of 1, 2, 3, 4)

\setlength\parindent{0pt} % Removes all indentation from paragraphs - comment this line for an assignment with lots of text

\newcommand{\horrule}[1]{\rule{\linewidth}{#1}} % Create horizontal rule command with 1 argument of height


%----------------------------------------------------------------------------------------
%	TÍTULO Y DATOS DEL ALUMNO
%----------------------------------------------------------------------------------------

\title{
\normalfont \normalsize
\textsc{{\bf Servidores web de altas prestaciones (2016-2017)} \\ Grado en Ingeniería Informática \\ Universidad de Granada} \\ [25pt] % Your university, school and/or department name(s)
\horrule{0.5pt} \\[0.4cm] % Thin top horizontal rule
\huge Práctica 2:  Clonar la información de un sitio web\\ % The assignment title
\horrule{2pt} \\[0.5cm] % Thick bottom horizontal rule
}

\author{Carlos Sánchez Martínez} % Nombre y apellidos

\date{\normalsize\today} % Incluye la fecha actual

%----------------------------------------------------------------------------------------
% DOCUMENTO
%----------------------------------------------------------------------------------------

\begin{document}

\maketitle % Muestra el Título

\newpage %inserta un salto de página

\tableofcontents % para generar el índice de contenidos

\listoffigures

\listoftables

\newpage




%----------------------------------------------------------------------------------------
%	Cuesti´on 1
%----------------------------------------------------------------------------------------

\section{ Probar el funcionamiento de la copia de archivos por ssh. }

	Para copiar el archivo .tar de la máquina 1 a la máquina dos usaremos el comando tar czf - \textit{directorio} | ssh \textit{dirección máquina 2} 'cat ~/\textit{directorio}.tar'.\\
  Para ello se ha realizado las siguientes capturas de pantalla.\\

  Primero se mostrará el directorio de la máquina 2 vacío.\\

  \begin{figure}[H]
		\centering
		\includegraphics[width=6in]{directorioMaquina2}
		\caption{Se muestra el directorio de la máquina 2 vacío.}
		\label{figura 1}
	\end{figure}

  Se muestra la ejecución del comando antes mencionado para copiar, desde máquina 1, el archivo prueba que se mostrará en la máquina 2 como prueba.tgz.\\

  \begin{figure}[H]
		\centering
		\includegraphics[]{copiaDirMaq1aMaq2}
		\caption{Se muestra la ejecución del comando para la copia del archivo prueba desde máquina 1 hasta máquina 2.}
		\label{figura 2}
	\end{figure}

  Se mostrará el archivo copiado en el directorio de la máquina 2.\\

  \begin{figure}[H]
		\centering
		\includegraphics[width=6in]{DirEnMaq2}
		\caption{Se muestra el archivo prueba copiado como prueba.tgz}
		\label{figura 3}
	\end{figure}


%--------------------------------------------------------------------------------------------------
%	Cuestion2
%--------------------------------------------------------------------------------------------------

\section{ Clonado de una carpeta entre las dos máquinas.}
	Para clonar un archivo desde una máquina a otra lo que se ha hecho ha sido clonar el directorio /var/www de la máquina 1 en la máquina 2. Para ello se ha utilizado la heramienta rsync y se ha utlizado el comando rsync -avz -e ssh \textit{dirección máquina1}:/var/www \textit{directorio destino}.\\
  El resultado de ejecutar el comando ha sido el siguiente:\\

  \begin{figure}[H]
		\centering
		\includegraphics[]{Ejer2}
		\caption{Se ha clonado el directorio /var/www de la máquina 1 en el directorio /var/www de la máquina 2.}
		\label{figura 4}
	\end{figure}

%----------------------------------------------------------------------------------------------
% Cuestion 3
%----------------------------------------------------------------------------------------------

\section{ Configuración de ssh para acceder sin que solicite contraseña. }
  Para acceder desde una máquina remota a otra sin contraseña con ssh se realizarán los siguientes pasos:\\
  1. Vamos a acceder a la máquina 1 desde la máquina 2 para ello primero ejecutaremos el comando ssh-keygen  -b 4096 -t rsa en la máquina 2.\\
  2. Ahora se habrá creado un archivo con una llave publica lo añadiremos al archivo ~/.ssh/authorized\_keys y se le añadirán los perimisos 600.\\
  3. Luego ejecutaremos el comando ssh-copy-id \textit{dirección máquina 1} para hacer la copia de la clave.\\
  4. Ejecutamos el comando ssh \textit{dirección máquina 1} para poder hacer cambios desde la máquina 2 en la máquina 1.\\
  Tras la ejecución de estos pasos y ejecutando el último paso ssh \textit{dirección de máquina 1} nos queda el resultado siguiente:\\

  \begin{figure}[H]
		\centering
		\includegraphics[]{Ejer3}
		\caption{Ejecución del comando ssh \textit{dirección máquina 1}. carlos\@ubuntu es el terminal de la máquina 1. (En este caso tiene el mismo nombre de la máquina 2.)}
		\label{figura 5}
	\end{figure}

  Para salir de la terminal de máquina 1 se ejecuta el comando exit.\\

%------------------------------------------------------------------------------------------------------
%	Cuestion 4
%------------------------------------------------------------------------------------------------------
\section{ Establecer una tarea en cron que se ejecute cada hora para mantener actualizado el contenido del directorio /var/www entre las dos máquinas. }
  Para crear una tarea vamos a modificar el archivo crontab y añadir la siguiente linea:\\
  0 * * * * rsync -avz -e ssh 10.0.2.5:/var/www /var/www  --report  /etc/cron.hourly
  \begin{figure}[H]
		\centering
		\includegraphics[]{crontab}
		\caption{Fichero /etc/crontab para actualizar /var/www cada hora a las en punto.}
		\label{figura 6}
	\end{figure}


\end{document}
